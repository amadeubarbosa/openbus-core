%
%  Documento de Instala��o do OpenBus 2.0
%
%  Created by Hugo Roenick on 2012-05-22.
%  Copyright (c) 2012 Tecgraf/PUC-Rio. All rights reserved.
%
\documentclass[]{article}

% Use utf-8 encoding for foreign characters
\usepackage[latin1]{inputenc}

\usepackage[brazil]{babel}

% Setup for fullpage use
\usepackage{fullpage}

% Uncomment some of the following if you use the features
%
% Running Headers and footers
%\usepackage{fancyhdr}

% Multipart figures
%\usepackage{subfigure}

% More symbols
%\usepackage{amsmath}
%\usepackage{amssymb}
%\usepackage{latexsym}

\usepackage{hyperref}

% Surround parts of graphics with box
\usepackage{boxedminipage}

% Package for including code in the document
\usepackage{../mwlabinputs2}

% If you want to generate a toc for each chapter (use with book)
\usepackage{minitoc}

% This is now the recommended way for checking for PDFLaTeX:
\usepackage{ifpdf}

%% Redefines the label 'Listing' for ..
\def\lstlistingname{C�digo}
\codestyle{colorful}

% new commands
\newcommand{\term}[1]{\textit{#1}}
\newcommand{\code}[1]{\texttt{#1}}

\newcommand{\openbus}{\textsc{OpenBus}}
\newcommand{\corba}{\textsc{CORBA}}
\newcommand{\orb}{\textsc{ORB}}
\newcommand{\scs}{\textsc{SCS}}
\newcommand{\lua}{\textsc{Lua}}
\newcommand{\oil}{\textsc{OiL}}
\newcommand{\version}{2.0.0}


%\newif\ifpdf
%\ifx\pdfoutput\undefined
%\pdffalse % we are not running PDFLaTeX
%\else
%\pdfoutput=1 % we are running PDFLaTeX
%\pdftrue
%\fi

\ifpdf
\usepackage[pdftex]{graphicx}
\else
\usepackage{graphicx}
\fi

\title{Manual de instala��o do \openbus{} \version{}}
\author{TecGraf}

\date{\today}

\begin{document}

\ifpdf
\DeclareGraphicsExtensions{.pdf, .jpg, .tif}
\else
\DeclareGraphicsExtensions{.eps, .jpg}
\fi

\maketitle

\tableofcontents

\section{Introdu��o}

Este documento visa informar apenas o procedimento necess�rio para instalar um barramento \openbus{}~\cite{web:OPENBUS} da vers�o \version{}.
Caso tenha interesse de entender melhor o que � um barramento \openbus{}, consulte o seu manual de refer�ncia~\cite{ob2.0core}.

\section{Instala��o}

O primeiro para instalar o barramento deve ser descompactar o pacote do barramento da plataforma desejada.
Os pacotes oficialmente suportados est�o disponibilizados no site oficial do projeto: \url{www.tecgraf.puc-rio.br/openbus}.
As plataformas oficialmente suportadas s�o:
\begin{itemize}
  \item SunOS510\_64
  \item SunOS510
  \item Linux26g4\_64
  \item Linux26g4
  \item Linux24g3\_64
  \item Linux24g3
\end{itemize}

Para executar o barramento deve-se seguir o seguinte procedimento:

\begin{enumerate}
  \item Extrair o pacote. Para fins de legibilidade, vamos guardar o caminho do local da extra��o em uma vari�vel de ambiente. Por�m, \textbf{essa declara��o n�o � obrigat�ria!}
\begin{verbatim}  
export OPENBUS_HOME=<local de extracao do pacote>
\end{verbatim}

  \item Configurar a vari�vel de ambiente LD\_LIBRARY\_PATH
\begin{verbatim}  
export LD_LIBRARY_PATH="${OPENBUS_HOME}/lib:${LD_LIBRARY_PATH}"
\end{verbatim}

  \item Gerar um novo par de chaves do barramento utilizando o script \emph{openssl-generate.ksh}.
\begin{verbatim}
cd ${OPENBUS_HOME} && ./specs/shell/openssl-generate.ksh -n <nome do par de chaves>
\end{verbatim}  
  
  \item Testar se a execu��o do bin�rio do barramento esta funcionando
\begin{verbatim}  
$OPENBUS_HOME/bin/busservices --help
\end{verbatim}

\end{enumerate}

Maiores informa��es sobre a como utilizar a configura��o do \emph{busservices} podem ser encontradas no manual de refer�ncia~\cite{ob2.0core}.

\section{FAQ}

\subsection{Como configurar o validador LDAP?}

Informa��es sobre como configurar o validado LDAP do barramento podem ser encontradas em \cite{ob2.0ldap}.

\subsection{Ainda estou com d�vidas. Como entro em contato?}

Disponibilizamos uma lista p�blica de discuss�o com o intuito de reconhecer os usu�rios da nossa tecnologia, bem como receber sugest�es e cr�ticas que contribuam para a evolu��o do nosso projeto.

Maiores informa��es sobre a nossa lista de discuss�o veja \url{http://listas.tecgraf.puc-rio.br/mailman/listinfo/openbus-users}.

\bibliographystyle{plain}
\bibliography{../references}

\end{document}
